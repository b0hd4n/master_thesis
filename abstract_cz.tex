%%% A template for a simple PDF/A file like a stand-alone abstract of the thesis.
\pdfminorversion=5

\documentclass[12pt]{report}

\usepackage[a4paper, hmargin=1in, vmargin=1in]{geometry}
\usepackage[a-2u]{pdfx}
\usepackage[utf8]{inputenc}
\usepackage[T1]{fontenc}
\usepackage{lmodern}
\usepackage{textcomp}

\begin{document}

%% Do not forget to edit abstract.xmpdata.

V mezinárodním a vysoce mnohojazyčném prostředí se často
stává, že řeč, dokument nebo jakýkoli jiný vstup musí být
přeložen do velkého počtu jazyků.

Není to vždy možné, mít zvlaštní systém pro každou možnou
dvojici jazyků vzhledem k tomu, že takový druh překladatelských
systémů je výpočetně náročný.

Kombinace více cílových jazyků do jednoho překladového modelu
obvykle způsobuje snížení kvality výstupu pro každý jeho směr
překladu.

V této práci provádime experimenty s kombinací cílových jazyků,
abychom zjistili, zda jejich nejaké konkrétní seskupení může vést
k lepším výsledkům, v porovnání s náhodným výběrem cílových jazyků.

Využíváme výsledky nejnovějších výzkumích prací o školení
vícejazyčného modelu transformátoru beze změny jeho architektury:
přidáváme značku cílového jazyku do zdrojové věty.

Natrénováli jsme řadu dvojjazyčných a vícejazyčných Transformer
modelů a vyhodnotili jsme je na několika testovacích sadách z
různých domén.
Zjistili jsme, že ve většině případů seskupení souvisejících
cílových jazyků do jednoho modelu způsobuje lepší výkon ve
srovnání s modely s náhodně vybranými jazyky.

Zjistili jsme však také, že doména testovací množiny, stejně jako
domény dat vybráných do trénovácí množiny, má obvykle významnější
vliv na zlepšení nebo zhoršení kvality překladu vícejazyčného modelu
ve srovnání s dvojjazyčnou.

\end{document}

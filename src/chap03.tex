\chapter{Random choise of target languages}


\section{Overview}

In this chapter we explore the effect of increasing number of target languages
on the model performance in general.


Multiple possible outcomes can be expected at this experiment:
either performance drop due to the increased amount of languages to be processed
by the model of the same size, or the opposite - performance increase due to
shared knowledge gained by the model from bigger and varying dataset.
Also, either of these options can be true for different target languages in different scale.

First of all, performance drop is expected.
Considering that the size of the model is fixed, so is its capacity.
At some moment adding more target languages should lead to the decrease
in translation quality for each of every target language


\section{Performance drop on massively multilingual setup}
1-to-3, 5, 7, etc. models on en-to-36 dataset (0.9 mil. sentences per target language)

When the size of the model is fixed, adding more translation directions usually causes
worsening of its performance. Multiple studies have shown this to be truth for
many-to-many setup.



\begin{table}[h!]
\begin{subtable}[t]{0.45\linewidth}
	\centering
	\begin{tabular}{rrrr}
	\toprule
	n\_targets &   mean &   std & count \\
	\midrule
	         1 &  41.40 &  ---  &   1 \\
	         2 &  40.60 &  0.20 &   3 \\
	         3 &  39.39 &  0.62 &   8 \\
	         4 &  39.40 &  0.71 &   2 \\
	         5 &  38.45 &  0.52 &   6 \\
	\bottomrule
	\end{tabular}

	\caption{
		En\to{}Bg for \emph{Europarl/v7} dataset.
		}
	\label{tab:bg/Europarl/v7}
\end{subtable}
\begin{subtable}[t]{0.45\linewidth}
	\centering
	\begin{tabular}{rrrrrrr}
	\toprule
	n\_targets & mean & count & std \\
	\midrule
	        1 &     19.50 &    1 &   --  \\
	        2 &     18.88 &    4 &  0.39 \\
	        3 &     17.45 &    4 &  0.52 \\
	        4 &     17.80 &    2 &  0.42 \\
	\bottomrule
	\end{tabular}
	
	\caption{
		En\to{}Ru for \emph{OpenSubtitles/v2016} dataset.
		}
	\label{ table:ru/OpenSubtitles/v2016 }
\end{subtable}
\mycaption{BLEU score change with adding target languages}{
    (a) First row: for mono-lingual En\to{}Bg model test BLEU score is 41.40.
    Second row: for 3 (column \emph{count}) En\to{}Any
    models with two target languages
    (column \emph{n\_targets}) one of which is Bulgarian
    the mean BLEU score is 40.60 with standard deviation 0.20.
    (b): same way as (a)
}
\end{table}



% \begin{table}[h]
% \centering
% \begin{tabular}{rrrrrrr}
% \toprule
% n\_targets & mean & count & std \\
% \midrule
%         1 &     --.-- &    1 &    -  \\
%         2 &     18.86 &    8 &  0.31 \\
%         3 &     17.59 &    8 &  0.48 \\
%         4 &     17.80 &    4 &  0.35 \\
% \bottomrule
% \end{tabular}
% 
% \caption{
% 	BLEU score for En\to{}Ru translation on test set part of
% 	\emph{OpenSubtitles/v2016} dataset.
% 	Description is the same as for table \ref{tab:bg/Europarl/v7}
% 	}
% \label{ table:ru/OpenSubtitles/v2016 }
% \end{table}

\section{Performance decrease on richer data sets}
1 to 3, 4, 5 on UN corpus (much more sentence pairs per target language)
\cite{eisele-chen-2010-multiun}



% https://www.overleaf.com/learn/latex/glossaries
% example:
% \acrshort{mt} - MT
% \acrlong{mt} - machine translation
% \acrfull{mt} - MT (machine translation)

\makeglossaries

\newglossaryentry{latex}
{
    name=latex,
    description={Is a mark up language specially suited 
    for scientific documents}
}

\newacronym{mt}{MT}{machine translation}
\newacronym{smt}{SMT}{statistical machine translation}
\newacronym{nmt}{NMT}{neural machine translation}
%%% \newacronym{}{}{}
%%% WMT18, WMT19, WMTxx -- annual Workshop on Statistical Machine Translation
%%% (year 2018, 2019, 20xx resp.)
\newacronym{rnn}{RNN}{recurrent neural network}
\newacronym{cnn}{CNN}{convolutional neural network}
\newacronym{lstm}{LSTM}{long short-term memory}
% --  (RNN architecture)
\newacronym{gru}{GRU}{gated recurrent unit}
\newacronym{alpac}{ALPAC}{Automatic Language Processing Advisory Committee}
\newacronym{arpa}{ARPA}{Advanced Research Projects Agency}
\newacronym{bleu}{BLEU}{bilingual evaluation understudy}
%--  (method of automatic MT evaluation)
%%% \newacronym{}{}{}
